%******************************************************************************%
%                                                                              %
%                  sample.fr.tex for LaTeX                                     %
%                  Created on : Tue Mar 10 13:27:28 2015                       %
%                  Made by : David "Thor" GIRON <thor@42.fr>                   %
%                                                                              %
%******************************************************************************%

\documentclass{42-fr}


%******************************************************************************%
%                                                                              %
%                                  Prologue                                    %
%                                                                              %
%******************************************************************************%
\begin{document}



                           \title{Cozy}
                          \subtitle{friends\_with\_benefits}
                         \member{ppeltier}{ppeltier@student.42.fr}
                      \member{42 Staff}{pedago@staff.42.fr}

\summary {
	L'open source c'est cool.
}

\maketitle

\tableofcontents


%******************************************************************************%
%                                                                              %
%                                  Préambule                                   %
%                                                                              %
%******************************************************************************%
\chapter{Préambule}


	\section{Ou sont mes donn\'ees ?}

		Aujourd'hui chaque personne produit une grande quantit\'ee de donn\'es
		qui se retrouvent dispers\'ees partout sur le web faisant la richesse de
		grandes entreprises. Ces informations peuvent etre aussi d'une grande
		utilit\'e pour l'utilisateur mais le probl\`eme est qu'\'etant completement
		\'eparpill\'e il devient tr\`es dure de les r\'ecuperer et d'en tirer
		avantage. L'autre probl\`eme est que c'est les h\'ebergeurs des services
		qui gardent vos informations donc vous n'avez pas la mains dessus.


	\section{And.... Cozy !}

		C'est pour r\'epondre \`a ces probl\`emes que Cozy a \'et\'e  cr\'e\'e,
		Le but est de vous aider \`a regrouper toutes vos donn\'ees en dans endroit
		qui vous appartient et donc qui vous conf\`ere controle total sur les
		donn\'es que vous h\'egergerez: traitement, modification, suppression.\\

		\subsection{Comment ca marche?}

			Cozy est un logiciel open-source que vous pouvez h\'egerger chez vous
			avec une base de donn\'ees regroupant toutes vos informations et une liste
			d'applications installable qui vous permet d'agir avec elle: cela vous
			permet entre autre de rassembler vos contacts, g\'erer vos
			informations banquaires ou encore de controller vos mails. Il y a de
			nombreuses autres applications d\'ej\`a existante comme les notes,
			la musique et autres.

	\section{Cool... et 42 dans tout ca ?}

		Cozy est open-source et fonctionne grace \`a des applications qui
		fonctionnent comme des modules. Elles sont d\'evolopp\'es par la
		team Cozy mais aussi par la communaut\'e. Le but de ce partenariat est
		de vous permettre via divers sujets d'avancer dans votre scholarit\'e
		tout en contribuant \'a la communaut\'e open-source et au projet Cozy.


%******************************************************************************%
%                                                                              %
%                                 Introduction                                 %
%                                                                              %
%******************************************************************************%
\chapter{Introduction}

	Ce projet vous propose de cr\'eer une application open-source de gestion de
	comptes entre amis \'evoluant dans l'environnement Cozy.

	\section{Gestion de comptes en amis: WTF?}

		C'est une application ou l'on peut tenir un historique des d\'epenses faites
		dans le cadre d'un \'ev\`eenement et qui indique a chaque participant
		quelle somme il doit rembourser a qui. Somme calcul\'ee en fonction
		de ses d\'epenses et des ses consommations.


	\section{Application Cozy ?}

		Ce sera une "web-app", ce qui veut indique qu'il ne doit pas y avoir
		de chargements. Toutes les modifications sont faites dans une seul et
		unique page ou le javascript s'occupera de d'afficher les diff\'erentes
		vues.\\
		Cette application doit etre utilisable dans une instance Cozy, ce qui
		l'oblige a s'adapter aux diff\'erents outils mise a disposition.
		Vous verrez assez vite que cela facilite le travail plus qu'autre chose
		car vous n'avez plus a vous occuper de la base de donn\'ee, des
		utilisateurs, ou m\^eme de l'installation.


	\section{Open-source?}

		Ce projet c'est aussi de l'open-source, ce qui implique que si vous voulez
		que l'on contribue a votre code il faut qu'il soit claire et document\'e.
		La communaut\'e d'utilisateur que vous allez monter autour de votre app
		peut vous remonter des erreurs ou m\^eme des souhaits, pensez \`a \'echanger
		avec elle.



%******************************************************************************%
%                                                                              %
%                                  Objectifs                                   %
%                                                                              %
%******************************************************************************%
\chapter{Objectifs}


	Ce sujet est une introduire au d\'eveloppement de web-apps ainsi qu'au
	d\'eveloppement de projet l'open-source. Cela vous demandera l'utilisation de
	differentes API, des templates ainsi que le framework de votre choix. Vous devrez
	faire du javascript cot\'e client et nodejs cot\'e serveur.
	Votre projet a pour but d'etre utili\'e et repris par d'autres
	d\'eveloppeurs, ce qui implique entre autre du code claire et de la communication
	avec les utilisateurs.

	\warn {
		Open-source ne veut pas dire pomper l'application du voisin et changer 5 lignes
		de codes...
	}

%******************************************************************************%
%                                                                              %
%                             Consignes generales                              %
%                                                                              %
%******************************************************************************%
\chapter{Consignes g\'en\'erales}


    \section{Environnement}
        L'app doit s'integrer a l'environnement Cozy, ce qui comprend:

        \begin{itemize}\itemsep1pt
            \item L'installation par lien Github a partir du MarketPlace
            \item L'utilisation des diff\'erentes outils/API
            \item L'utilisation du data-system
            \item L'application doit etre \textbf{au moins} anglophone
        \end{itemize}

    \section{L'application}
        \begin{itemize}\itemsep1pt
            \item L'application doit etre en one-page, ce qui induit que le
				changement de contenu est dynamique et se fait dans la meme page
            \item Une gestion dynamique de l'url doit etre faite: l'url change
				suivant le contenus que j'affiche et l'on dois avoir la
        possibilit\'e de charger directement ce contenue si on lance cette url.
        \end{itemize}

    \section{Language/Outil}
        \begin{itemize}\itemsep1pt
            \item Le language cote client est Javascript (ECMAScript 5 ou 6)
            \item Le language cote server est Nodejs (la version support\'e)
            \item Tout code doit etre maintenable, document\'e et comment\'e
            \item Le choix des differents outils/frameworks est libre
        \end{itemize}

		\hint {
			Des tutos et outils sont present sur le site de Cozy
		}

	\section{Open-source}
		Votre application doit etre open-source et doit donc posseder une license ad\'equate
		ainsi que les outils ou librairies que vous utiliserez.



%******************************************************************************%
%                                                                              %
%                             Partie obligatoire                               %
%                                                                              %
%******************************************************************************%
\chapter{Partie obligatoire}


    \section{Fonctionnalit\'es g\'en\'erales}

        L'application doit pouvoir au moins:\\

        \begin{itemize}\itemsep1pt
            \item Afficher la liste des comptes
			\item G\'erer les comptes (ajout/suppression/modification/...)
            \item Avoir un gestion d'utilisateurs global ou interne a chaque compte (au choix)
			\item Afficher le contenus de chaque compte
			\item Afficher l'ensemble des transactions a faire entre les
				utilisateurs pour qu'ils puissent \'equilibrer leur d\'epenses
			\item Etre responsive sur tout les supports
        \end{itemize}


    \section{Fonctionnalit\'es d'un compte}

        La vue d'un compte doit pouvoir au moins:\\

        \begin{itemize}\itemsep1pt
            \item Afficher les utilisateurs participants
            \item Afficher l'historique des d\'epenses du compte
			\item Ajouter une d\'epense
			\item Afficher l'\'etat actuel du compte avec par exemple: la somme
				de toutes les d\'epenses, le nombre de d\'epenses, des graphes,
				pleins d'autres stats utiles.
        \end{itemize}


    \section{Ergonomie}

        L'ergonomie g\'en\'eral sera prise en compte dans la correction,
		ceci n'inclue pas le graphisme/design: votre site ne doit pas necessairement
		etre 'beau', juste pratique.


%******************************************************************************%
%                                                                              %
%                                 Partie bonus                                 %
%                                                                              %
%******************************************************************************%
\chapter{Partie bonus}

    Bonus possible:\newline

    \begin{itemize}\itemsep1pt
		\item Localisation (languages)
        \item Une partie public pour que tout les utilisateurs puisse acceder a
			l'app sans connexion au compte Cozy
        \item Graphisme/design pouss\'ee
        \item Gestion des devises
		\item Real-time
		\item Gestion des emprunts/prets entre utilisateurs
		\item Utilisation des donn\'ees et avatars de l'application Contact
    \end{itemize}

    \hint {
        Vous avez pas d'id\'ees de bonus? Demandez a la communaut\'e !
    }



%******************************************************************************%
%                                                                              %
%                           Rendu et peer-evaluation                           %
%                                                                              %
%******************************************************************************%
\chapter{Rendu et peer-\'evaluation}

	Travaillez sur un depot github en public pour que les gens puissent suivrent
	votre travail et pouvoir pull votre travail une fois terminer, mais pour la
	correction rendez votre travail sur votre d\'epot \texttt{GiT} comme d'habitude.
	Seul le travail pr\'esent sur votre d\'epot priv\'e sera \'evalu\'e en soutenance.\\
	Les consignes de la partie g\'en\'erales et obligatoire doivent etre faites
	mais sont soumises \`a votre interpretation, ce sera \`a vous de justifier
	\`a vos correcteur votre maniere d'implementer et ses raisons.




%******************************************************************************%
\end{document}
