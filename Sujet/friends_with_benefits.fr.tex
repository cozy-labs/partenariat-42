%******************************************************************************%
%                                                                              %
%                  sample.fr.tex for LaTeX                                     %
%                  Created on : Tue Mar 10 13:27:28 2015                       %
%                  Made by : David "Thor" GIRON <thor@42.fr>                   %
%                                                                              %
%******************************************************************************%

\documentclass{42-fr}


%******************************************************************************%
%                                                                              %
%                                  Prologue                                    %
%                                                                              %
%******************************************************************************%
\begin{document}



                           \title{Cozy Cloud}
                          \subtitle{friends\_with\_benefits}
                         \member{ppeltier}{ppeltier@student.42.fr}
                      \member{42 Staff}{pedago@staff.42.fr}

\summary {
	Cr\'eation d'une application web de gestion de comptes entre amis pour la plateforme de Cloud Personnel open-source, Cozy Cloud.
}

\maketitle

\tableofcontents


%******************************************************************************%
%                                                                              %
%                                  Préambule                                   %
%                                                                              %
%******************************************************************************%
\chapter{Préambule}


	\section{O\`u sont mes donn\'ees ?}

		Que vous le sachiez ou non, vous produisez chaque jour une grande quantit\'e de donn\'ees
		qui sont capt\'ees, enregistr\'ees, et utilis\'ees partout sur le web, faisant
		la richesse des entreprises qui se les accaparent. Vous pourriez, vous aussi,
		tirer la juste valeur de vos donn\'ees, mais ces derni\`eres sont \'eparpill\'ees
		chez tous les acteurs, qui souvent cherchent \`a ne surtout pas les partager.
		Vous n'avez plus le contrôle de vos donn\'ees, n'en \^etes plus les propri\'etaires.
		Alors, que faire ?


	\section{And…Cozy Cloud !}

		C'est pour r\'epondre \`a cette probl\'ematique que Cozy Cloud a \'et\'e  cr\'e\'e,
		R\'eappropriez-vous vos donn\'ees \`a travers une plateforme open-source, que vous contr\^olez
		compl\`etement. Centralisez un maximum vos donn\'ees et installez des services
		qui les utiliseront de façon respectueuse. D\'ecouvrez un cloud vraiment personnel !

		\subsection{Comment ça marche ?}

			Cozy Cloud est un logiciel open-source que vous pouvez h\'eberger chez vous,
			qui vous permet de rassembler le plus grand nombre de donn\'ees
			personnelles possible. Ces donn\'ees peuvent ensuite \^etre utilis\'ees par des
			applications installables dans le but de fournir un service. Par exemple,
			rassembler vos contacts, g\'erer votre agenda, partager vos fichiers et
			photos, consulter vos emails ou encore conna\^itre l'\'etat de vos comptes
			bancaires. Il y a environ 25 applications de toute sorte : lecteur de musique,
			gestion de notes, etc.

	\section{Cool… et 42 dans tout ca ?}

		Cozy Cloud permet le d\'eploiement d'applications qui fonctionnent comme
		des modules. Pensez \`a Android, mais pour un serveur. Les applications de
		base sont sont d\'evolopp\'ees par l'\'equipe de Cozy Cloud, mais toutes
		les autres le sont par la communaut\'e. Le but de ce partenariat est de vous
		permettre de d\'evelopper vos comp\'etences en d\'eveloppement web tout
		en contribuant \`a la communaut\'e et au projet Cozy Cloud.


%******************************************************************************%
%                                                                              %
%                                 Introduction                                 %
%                                                                              %
%******************************************************************************%
\chapter{Introduction}

	Ce projet vous propose de cr\'eer une application web open-source de gestion de
	comptes entre amis pour la plateforme Cozy Cloud.

	\section{Gestion de comptes en amis : WTF ?}

		Vous avez d\'ej\`a organisé une soir\'ee \`a plusieurs ? Vous \^etes d\'ej\`a partis en
		vacances avec des amis ? Pas facile de savoir qui paie quoi, et qui doit
		combien \`a qui. Cette application a pour but d'\'eviter ce casse-t\^ete. Remplissez
		l'historique des d\'epenses, qui a pay\'e, qui participe et l'application
		indiquera quel participant doit combien \`a quel autre participant.

	\section{Application Cozy Cloud ?}

		Une application Cozy Cloud est une application web un peu particuli\`ere.
		Cozy Cloud \'etant une plateforme, vous n'aurez pas \`a vous soucier du
		d\'eploiement, de la gestion de l'authentification ou encore de la base de
		donn\'ees. Vous serez donc amener \`a d\'ecouvrir la plateforme elle-m\^eme pour
		vous faciliter le plus possible la vie.


	\section{Open-source ?}

		Ce projet est pour vous l'occasion d'\'ecrire du code open-source.
		Cela implique des contributions, sous forme de remont\'es de bugs, des
		demandes de fonctionnalit\'es ou m\^eme du code produit par un d\'eveloppeur
		ext\'erieur \`a votre projet !
		Confront\'e \`a la plateforme, vous aurez certainement des remarques \`a
		adresser \`a ses d\'eveloppeurs : n'h\'esitez pas et contribuer aussi au projet !



%******************************************************************************%
%                                                                              %
%                                  Objectifs                                   %
%                                                                              %
%******************************************************************************%
\chapter{Objectifs}


	Ce sujet est une introduction au d\'eveloppement d'applications web dans un contexte
	open-source. Vous serez amen\'e \`a choisir et utiliser diff\'erentes APIs,
	biblioth\`eques et technologies. Vous devrez faire du javascript c\^ot\'e
	client et Node.js c\^ot\'e serveur. Votre projet a pour but d'\^etre utilis\'e
	par les utilisateurs de Cozy Cloud, voir d'\^etre repris par d'autres
	d\'eveloppeurs, ce qui implique de placer les futurs utilisateurs au centre
	de la conception de votre application, tout en pensant aux d\'eveloppeurs qui
	r\'eutiliseront votre code.

	\warn {
		Open-source ne veut pas dire pomper l'application du voisin et changer 5 lignes
		de codes…
	}

%******************************************************************************%
%                                                                              %
%                             Consignes generales                              %
%                                                                              %
%******************************************************************************%
\chapter{Consignes g\'en\'erales}


    \section{Environnement}
        L'application doit s'int\'egrer \`a l'environnement Cozy, c'est-\`a-dire :

        \begin{itemize}\itemsep1pt
            \item Elle sera installable via l'interface de Cozy Cloud.
            \item Elle utilisera la base de donn\'ees de Cozy Cloud.
            \item Elle sera \textbf{au moins} traduite en anglais.
        \end{itemize}

    \section{L'application}
        \begin{itemize}\itemsep1pt
            \item L'application devra \^etre une Single Page Application :
				le changement de contenu est dynamique au sein d'une page unique.
            \item Chaque \'ecran de l'application doit cependant poss\'eder une URL propre.
        \end{itemize}

    \section{Language/Outil}
        \begin{itemize}\itemsep1pt
            \item Le language c\^ot\'e client est Javascript (ECMAScript 5 ou 6).
            \item L'environnement c\^ot\'e serveur est Node.js.
            \item Tout code doit \^etre maintenable, document\'e et comment\'e.
            \item Le choix des diff\'erents outils et frameworks est libre.
        \end{itemize}

		\hint {
			Des tutoriels et outils sont pr\'esent sur le site de Cozy Cloud : http://cozy.io/.
			Soyez attentif \`a la version de Node.js de la plateforme, qui n'est
			pas la plus r\'ecente !
		}

	\section{Open-source}
		Votre application sera open-source et devra poss\'eder une license ad\'equate.
		Attention aussi aux outils et biblioth\`eques que vous utiliserez !



%******************************************************************************%
%                                                                              %
%                             Partie obligatoire                               %
%                                                                              %
%******************************************************************************%
\chapter{Partie obligatoire}


    \section{Fonctionnalit\'es g\'en\'erales}

        L'application doit pouvoir a minima :\\

        \begin{itemize}\itemsep1pt
            \item Afficher la liste des comptes ;
			\item G\'erer les comptes (ajout / suppression / modification) ;
            \item G\'erer les participants d'un compte ;
			\item Afficher le r\'esum\'e de chaque compte ;
			\item \^Etre utilisable sur diff\'erents supports (pensez \`a l'usage mobile !).
        \end{itemize}


    \section{Fonctionnalit\'es d'un compte}

        Le r\'esum\'e d'un compte poss\`edera au moins les fonctionnalit\'es suivantes :\\

        \begin{itemize}\itemsep1pt
            \item Afficher les participants au compte ;
            \item Afficher l'historique des d\'epenses du compte ;
			\item Ajouter une d\'epense ;
			\item Afficher la balance du compte : quel participant doit combien \`a qui ;
	        \item Une partie publique \`a laquelle les participants peuvent acc\'eder sans
				\^etre connect\'e au Cozy.
        \end{itemize}


    \section{Ergonomie}

        L'ergonomie g\'en\'eral sera prise en compte dans la correction,
		ceci n'inclut pas le graphisme/design : votre application ne doit
		pas n\'ecessairement \^etre belle, mais pratique et intuitive.

		\hint {
			L'ergonomie recoupe \`a la fois la disposition g\'en\'erale des diff\'erents
			\'ecrans, les retours visuels \`a l'utilisateur, la position des boutons,
			la validation des formulaires, les raccourcis claviers, etc.
		}



%******************************************************************************%
%                                                                              %
%                                 Partie bonus                                 %
%                                                                              %
%******************************************************************************%
\chapter{Partie bonus}

    Si vous voulez enrichir votre application, voici quelques id\'ees :\newline

    \begin{itemize}\itemsep1pt
		\item G\'erer la localisation (plusieurs langues) ;
        \item Gestion des devises ;
		\item Afficher les changements en temps r\'eel (particuli\`erement utile si
			un participant ajoute une d\'epense depuis la partie publique) ;
		\item Afficher l'\'etat du compte sous forme graphique et statistiques ;
		\item Utiliser les fiches contacts du Cozy pour identifier les participants.
    \end{itemize}

    \hint {
        Vous avez pas d'id\'ees de bonus ? Demandez \`a la communaut\'e !
    }



%******************************************************************************%
%                                                                              %
%                           Rendu et peer-evaluation                           %
%                                                                              %
%******************************************************************************%
\chapter{Rendu et peer-\'evaluation}

	Vous travaillerez sur un d\'ep\^ot Github public pour que tout le monde puisse suivre
	votre travail et pouvoir facilement le r\'ecup\'erer une fois termin\'e. Cela permettra
	aussi \`a l'application d'\^etre installable par une instance Cozy Cloud.\newline
	En ce qui concerne la correction, vous d\'eposerez votre travail sur votre d\'ep\^ot
	\texttt{Git} comme d'habitude.\newline
	Seul le travail pr\'esent sur votre d\'ep\^ot priv\'e sera \'evalu\'e en soutenance.\\
	Les fonctionnalit\'es d\'ecrites dans les consignes g\'en\'erales et et la partie
	obligatoire doivent \^etre impl\'ement\'ees, mais il n'y a pas un unique rendu attendu :
	soyez cr\'eatif et en mesure de justifier vos choix fonctionnels et techniques \`a vos
	correcteurs.



%******************************************************************************%
\end{document}
