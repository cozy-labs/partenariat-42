%******************************************************************************%
%                                                                              %
%                  sample.tex for LaTeX                                        %
%                  Created on : Tue Mar 10 13:27:28 2015                       %
%                  Made by : David "Thor" GIRON <thor@42.fr>                   %
%                                                                              %
%******************************************************************************%

\documentclass{42}


%******************************************************************************%
%                                                                              %
%                                  Prologue                                    %
%                                                                              %
%******************************************************************************%
\begin{document}



                           \title{cozy\_count}
                          \subtitle{Your first open-source project}
                        \member{Cozy}{contact@cozy.cc}

\summary {
  Ce document est un exemple d'utilisation de LaTeX et de la
  classe \href{www.42.fr}{42}.
}

\maketitle

\tableofcontents


%******************************************************************************%
%                                                                              %
%                                  Préambule                                   %
%                                                                              %
%******************************************************************************%
\chapter{Preamble}

    \section{Qu'est ce que Cozy}
        Cozy est un cloud personnel open-source que vous pouvez heberg\'e et
        personnalis\'e qui vous permet une gestion de contact, calendrier, mails,
        etc tout au meme endroit afin d'en avoir le controle complet.


%******************************************************************************%
%                                                                              %
%                                 Introduction                                 %
%                                                                              %
%******************************************************************************%
\chapter{Introduction}

    L'introduction est une pr\'esentation des grandes lignes du
    projet. Il est appr\'eci\'e de donner un peu de contexte et une
    id\'ee du travail \`a r\'ealiser. Ainsi en lisant ces quelques
    lignes, un \'etudiant peut avoir une vue d'ensemble des th\`emes
    abord\'es.\newline


    \section{Resum\'e}
        Explication du projet:\newline

        Creation d'une web-app permettant la gestion de depenses pour des projets
        entre amis. Chaque utilisateurs peut enregistrer une depense qui sera
        redistribuer aux autres membres du groupe afin de pouvoir faire un partage
        equitable des couts du projet. Le projet a pour but d'aller sur le
        marketplace de Cozy et de recevoir des contributions.

    \section{Exemple}
        3 Users vont en vacances, 1 et 2 font 200euro de depenses ....\newline


%******************************************************************************%
%                                                                              %
%                                  Objectifs                                   %
%                                                                              %
%******************************************************************************%
\chapter{Objectifs}

    Il s'agit i\c ci d'expliquer l'interet p\'edagogique du projet,
    car au dela de la forme de celui-ci, un projet est avant tout un
    pr\'etexte \`a la d\'ecouverte ou \`a l'approfondissement d'une
    notion ou d'un groupe de notions. Par exemple le projet
    \texttt{Nibbler} : sous l'apparence d'un simple jeu
    \texttt{Snake}, ce projet permet d'initier les \'etudiant \`a la
    cr\'eation d'API et de plugins pour un programme en \texttt{C++}.\newline


    Le but est de vous introduire aux projets open-sources, aux web-app ainsi
    qu'aux differentes techno web tel que javascript, nodejs et le framework que
    vous choisirez.


%******************************************************************************%
%                                                                              %
%                             Consignes generales                              %
%                                                                              %
%******************************************************************************%
\chapter{Consignes g\'en\'erales}

    Cette section regroupe les consignes de base d'un
    projet. Langages, restrictions, autorisations, compilation,
    etc. En gros toutes les consignes ``meta''.\newline

    \section{Environnement}
        L'app doit s'integrer a l'environnement Cozy, ce qui comprend:

        \begin{itemize}\itemsep1pt
            \item L'installation par lien Github a partir du MarketPlace
            \item L'utilisation des diff\'erentes API
            \item L'utilisation du data-system
            \item L'application doit etre \textbf{au moins} anglophone
        \end{itemize}

    \section{L'application}
        \begin{itemize}\itemsep1pt
            \item L'app doit etre en one-page, ce qui induit qu'il n'y a aucuns
        changements de page.

            \item Une gestion dynamique de l'url doit cependant etre
        fait: l'url change suivant le contenus que j'affiche et je dois avoir la
        possibilit\'e de charger directement ce contenue si je lance cette url.
        \end{itemize}

    \section{Language/Outil}
        \begin{itemize}\itemsep1pt
            \item Le language cote client est Javascript (ECMAScript 5 ou 6)
            \item Le language cote server est Nodejs (la version supporte)
            \item Tout code doit etre maintenable, documenter et commenter
            \item Le choix des differents outils/framework est libre
        \end{itemize}


        \hint {
            Des tutos et outils sont present sur le site de Cozy
        }


%******************************************************************************%
%                                                                              %
%                             Partie obligatoire                               %
%                                                                              %
%******************************************************************************%
\chapter{Partie obligatoire}

    Partie centrale d'un sujet, la partie obligatoire d\'ecrit en
    d\'etails le travail \`a r\'ealiser et les conditions
    impose\'ees. Tout le secret d'un bon sujet r\'eside dans
    l'\'equilibre subtile entre \^etre pr\'ecis et laisser une part
    \`a l'interpr\'etation et a l'imagination. Ce facteur est
    important puisque c'est le moteur des discussions et des
    confrontations lors des soutenances.\newline


    \section{Compte}

        Un compte doit avoir:

        \begin{itemize}\itemsep1pt
            \item Une gestion des utilisateurs
            \item L'affichage de la liste des depenses
            \item Une gestion des depenses
            \item Un resume de la situation du compte (graphique/statistiques/...)
            \item Une section indiquant qui doit quelle somme a qui pour
            egaliser les depenses
        \end{itemize}


    \section{Depense}

        Chaque depense doit avoir:

        \begin{itemize}\itemsep1pt
            \item Un payeur
            \item Une liste de personnes impliquer dans le remboursement de
            l'achat
            \item Un montant
        \end{itemize}


    \section{Ergonomie}

        L'ergonomie general sera pris en compte dans la correction.
        L'application se doit d'etre facile a utiliser, intuitif et fluid. La
        responsive pour tout les supports est aussi demand\'e. Ceci n'inclue pas
        le graphisme/design: votre site ne doit pas necessairement etre 'beau',
        juste pratique.


%******************************************************************************%
%                                                                              %
%                                 Partie bonus                                 %
%                                                                              %
%******************************************************************************%
\chapter{Partie bonus}

    Lorsqu'on a investi du temps sur un projet et que le r\'esultat
    est au rendez-vous, il est naturel d'avoir envie d'aller plus loin
    ! La section bonus propose des ouvertures pour r\'epondre \`a
    cette envie. Bien entendu, la partie bonus n'est accessible que si
    et seulement si la partie obligatoire a \'et\'e r\'ealis\'ee
    enti\`erement et parfaitement.\newline

    Bonus possibles:\newline

    \begin{itemize}\itemsep1pt
        \item Localisation
        \item Une partie public pour que tout les utilisateurs y puisse y
        acceder.
        \item Graphisme/design pouss\'ee
        \item Gestion des devises
        \item ...
    \end{itemize}

    \hint {
        Vous avez pas d'id\'ees de bonus? Demand\'e a la communaut\'e !
    }


%******************************************************************************%
%                                                                              %
%                           Rendu et peer-evaluation                           %
%                                                                              %
%******************************************************************************%
\chapter{Rendu et peer-\'evaluation}

    Cette section d\'ecrit les conditions et les instructions
    concernant le rendu et la peer-\'evaluation du projet.\newline


    Il  vous est demand\'e de cr\'eer un application utilisable en production
    donc la correction se fera des ces conditions. L'application doit etre
    present sur le marketplace ou installable via le lien github et lancable
    depuis la page d'accueil.



\end{document}
%******************************************************************************%
