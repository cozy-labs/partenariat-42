%******************************************************************************%
%                                                                              %
%                  sample.tex for LaTeX                                        %
%                  Created on : Tue Mar 10 13:27:28 2015                       %
%                  Made by : David "Thor" GIRON <thor@42.fr>                   %
%                                                                              %
%******************************************************************************%

\documentclass{42}


%******************************************************************************%
%                                                                              %
%                                  Prologue                                    %
%                                                                              %
%******************************************************************************%
\begin{minted}{document}



                           \title{Document d'exemple}
                          \subtitle{Sous-titre de fou}
                        \member{Staff}{staff@staff.42.fr}

\summary {
  Ce document est un exemple d'utilisation de LaTeX et de la
  classe \href{www.42.fr}{42}.
}

\maketitle

\tableofcontents


%******************************************************************************%
%                                                                              %
%                                  Préambule                                   %
%                                                                              %
%******************************************************************************%
\chapter{Préambule}

    Le pr\'eambule d'un sujet est g\'en\'eralement sans rapport avec
    le sujet en lui-m\^eme. Il s'agit i\c ci de partager un trait
    d'humour (souvent discutable) ou une curiosit\'e avec la
    communaut\'e.\\

    Voici quelques \texttt{commandes} \textbf{pratiques} :\\

    \begin{itemize}\itemsep1pt
        \item Ho,
        \item la
        \item belle
        \item liste.
    \end{itemize}

    \begin{description}\itemsep3pt
        \item [Orange :] Fruit rond et orange.
        \item [Fraise :] Fruit en forme de fraise et rouge.
        \item [Comcombre :] L\'egume de forme phallique et vert.
    \end{description}

    \begin{enumerate}\itemsep7pt
        \item Un.
        \item Deux.
        \item Trois.
    \end{enumerate}

    \warn {
      Attention !
    }

    \hint {
      Le saviez-vous ?
    }

    \info{
      Pour info.
    }

    \begin{42console}
$sudo rm -rf /\end{42console}


    \begin{42ccode}
int main( void ) {

    puts( "hello world !" );
    return 0;
}
\end{42ccode}


    \begin{42cppcode}
int main( void ) {

    std::cout << "hello world !" << std::endl;
    return 0;
}
\end{42cppcode}

    \section{Exemple de section}

        \subsection{Exemple de sous-section}

            Message passionnant.\\

            Underscore : \_\\

            Et commercial : \&



%******************************************************************************%
%                                                                              %
%                                 Introduction                                 %
%                                                                              %
%******************************************************************************%
\chapter{Introduction}

    L'introduction est une pr\'esentation des grandes lignes du
    projet. Il est appr\'eci\'e de donner un peu de contexte et une
    id\'ee du travail \`a r\'ealiser. Ainsi en lisant ces quelques
    lignes, un \'etudiant peut avoir une vue d'ensemble des th\`emes
    abord\'es.



%******************************************************************************%
%                                                                              %
%                                  Objectifs                                   %
%                                                                              %
%******************************************************************************%
\chapter{Objectifs}

    Il s'agit i\c ci d'expliquer l'interet p\'edagogique du projet,
    car au dela de la forme de celui-ci, un projet est avant tout un
    pr\'etexte \`a la d\'ecouverte ou \`a l'approfondissement d'une
    notion ou d'un groupe de notions. Par exemple le projet
    \texttt{Nibbler} : sous l'apparence d'un simple jeu
    \texttt{Snake}, ce projet permet d'initier les \'etudiant \`a la
    cr\'eation d'API et de plugins pour un programme en \texttt{C++}.



%******************************************************************************%
%                                                                              %
%                             Consignes generales                              %
%                                                                              %
%******************************************************************************%
\chapter{Consignes g\'en\'erales}

    Cette section regroupe les consignes de base d'un
    projet. Langages, restrictions, autorisations, compilation,
    etc. En gros toutes les consignes ``meta''.



%******************************************************************************%
%                                                                              %
%                             Partie obligatoire                               %
%                                                                              %
%******************************************************************************%
\chapter{Partie obligatoire}

    Partie centrale d'un sujet, la partie obligatoire d\'ecrit en
    d\'etails le travail \`a r\'ealiser et les conditions
    impose\'ees. Tout le secret d'un bon sujet r\'eside dans
    l'\'equilibre subtile entre \^etre pr\'ecis et laisser une part
    \`a l'interpr\'etation et a l'imagination. Ce facteur est
    important puisque c'est le moteur des discussions et des
    confrontations lors des soutenances.



%******************************************************************************%
%                                                                              %
%                                 Partie bonus                                 %
%                                                                              %
%******************************************************************************%
\chapter{Partie bonus}

    Lorsqu'on a investi du temps sur un projet et que le r\'esultat
    est au rendez-vous, il est naturel d'avoir envie d'aller plus loin
    ! La section bonus propose des ouvertures pour r\'epondre \`a
    cette envie. Bien entendu, la partie bonus n'est accessible que si
    et seulement si la partie obligatoire a \'et\'e r\'ealis\'ee
    enti\`erement et parfaitement.


%******************************************************************************%
%                                                                              %
%                           Rendu et peer-evaluation                           %
%                                                                              %
%******************************************************************************%
\chapter{Rendu et peer-\'evaluation}

    Cette section d\'ecrit les conditions et les instructions
    concernant le rendu et la peer-\'evaluation du projet.



\end{document}
%******************************************************************************%
